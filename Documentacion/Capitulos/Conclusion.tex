El proyecto de Mini Plataforma de Video Streaming P2P con Microservicios ha cumplido exitosamente con todos los objetivos planteados:

\begin{itemize}
    \item \textbf{Arquitectura Distribuida:} Se implementó un sistema híbrido que combina un servicio central de registro con comunicación P2P directa entre nodos
    \item \textbf{Escalabilidad:} El sistema permite agregar nodos dinámicamente usando Docker Compose con escalado horizontal
    \item \textbf{Comunicación Eficiente:} Se estableció un sistema de notificaciones en tiempo real usando Redis Pub/Sub
    \item \textbf{Transferencia de Fragmentos:} Los nodos pueden intercambiar fragmentos de video de manera directa y eficiente
\end{itemize}

\subsection*{Beneficios del Enfoque P2P}

La implementación P2P demostró ventajas significativas sobre modelos centralizados tradicionales:

\begin{itemize}
    \item \textbf{Reducción de Carga Central:} El servidor central solo maneja registro y coordinación, no transferencia de datos
    \item \textbf{Tolerancia a Fallos:} Los fragmentos pueden estar replicados en múltiples nodos
    \item \textbf{Escalabilidad Natural:} Más nodos significan más capacidad total del sistema
\end{itemize}


\subsubsection*{Microservicios con Spring Boot}
La arquitectura de microservicios permitió:
\begin{itemize}
    \item Desarrollo independiente de componentes
    \item Mantenimiento simplificado
\end{itemize}

\subsubsection*{Containerización con Docker}
El uso de Docker facilitó:
\begin{itemize}
    \item Despliegue consistente en diferentes entornos
    \item Escalado automático de nodos
    \item Aislamiento de dependencias
    \item Configuración simplificada
\end{itemize}

\subsubsection*{Sistema Pub/Sub}
Redis Pub/Sub proporcionó:
\begin{itemize}
    \item Notificaciones en tiempo real
    \item Desacoplamiento entre componentes
    \item Sincronización de estado distribuido
\end{itemize}

\subsection*{Desafíos Superados}

Durante el desarrollo se enfrentaron y resolvieron varios desafíos técnicos:

\begin{itemize}
    \item \textbf{Sincronización de Estado:} Implementación de un sistema robusto de notificaciones para mantener consistencia
    \item \textbf{Transferencia de Archivos:} Desarrollo de un mecanismo eficiente para transferir fragmentos entre nodos
    \item \textbf{Auto-registro:} Configuración automática de nodos al iniciar usando variables de entorno
    \item \textbf{Manejo de Errores:} Implementación de tolerancia a fallos en comunicaciones de red
\end{itemize}



Este proyecto proporcionó experiencia práctica en:

\begin{itemize}
    \item Desarrollo de sistemas distribuidos
    \item Arquitectura de microservicios
    \item Comunicación P2P
    \item Containerización y orquestación
    \item Trabajo colaborativo en equipo
    \item Metodologías ágiles de desarrollo
\end{itemize}

